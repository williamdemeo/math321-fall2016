% Example LaTeX document for GP111 - note % sign indicates a comment
\documentclass[12pt,reqno]{amsart}
\usepackage[top=1.5cm, left=1.5cm,right=1.5cm,bottom=1.5cm]{geometry}
\usepackage{tikz}
\usepackage{amsmath}
\usepackage{amssymb}
\usepackage{mathrsfs}
%\usepackage{eucal}
\usepackage{color,hyperref,enumerate}
\definecolor{darkblue}{rgb}{0.0,0.0,0.3}
\hypersetup{colorlinks,breaklinks,
            linkcolor=darkblue,urlcolor=darkblue,
            anchorcolor=darkblue,citecolor=darkblue}
            
\usepackage{algorithm}
\usepackage{algorithmic}
\pagestyle{empty}
\newcommand{\Z}{\ensuremath{\mathbb{Z}}}
\newcommand{\N}{\ensuremath{\mathbb{N}}}
\newcommand{\R}{\ensuremath{\mathbb{R}}}
\newcommand{\sB}{\ensuremath{\mathscr{B}}}
\newcommand{\meet}{\ensuremath{\wedge}}
\newcommand{\Meet}{\ensuremath{\bigwedge}}
\newcommand{\mr}[1]{\ensuremath{\mathrel{#1}}}
\newcommand{\lra}{\ensuremath{\longleftrightarrow}}
\newcommand{\refine}{\ensuremath{\preccurlyeq}}
\newcommand{\join}{\ensuremath{\vee}}
\renewcommand{\emptyset}{\ensuremath{\varnothing}}
%\renewcommand{\subset}{\ensuremath{\subsetneq}}

\begin{document}
\thispagestyle{empty}

\begin{center}
\textbf{Remarks on Subset Notation}
\end{center}

\noindent $A\subseteq B$ means that $A$ is a subset of $B$.  That is,
\[
(\forall x)(x\in A \longrightarrow x\in B)
\]
Notice that it is possible for $A$ to be equal to $B$ in this case.
\\\\
Some authors use the notation $A\subset B$ to mean that $A$ is a 
subset of $B$ and $A\neq B$.  That is, both
\[
(\forall x)(x\in A \longrightarrow x\in B)\; \text{ \emph{and} }\; (\exists y)
(y\in B \meet y\notin A ).
\]
\\
In other words, $A\subset B$ means that $A$ is \emph{strictly} contained in $B$,
so there is at least one element in $B$ that is not in $A$; whereas
$A\subseteq B$ means that $A$ is contained in \emph{or equal to} $B$.
\\\\
This notation is completely analogous to notation you are already familiar with
involving inequality.  $A\leq B$ means that $A$ is less than or equal to $B$,
whereas $A< B$ means that $A$ is \emph{strictly} less than $B$.
\\\\
Unfortunately, there is at least one well known book that uses $\subset$ to mean ``subset or equal.''  So, personally, I never write $A\subset B$ because of the 
confusion that this inconsistent usage has caused.
Instead, on the rare occasions when I want to emphasize that a set is not only 
a subset, but also a \emph{proper} subset, then I might use $A\subsetneqq B$.
This means \emph{exactly the same thing} as
what some authors denote by $A\subset B$.
\\\\
To summarize $A\subsetneqq B$ is equivalent to $(A\subseteq B) \wedge (A \neq B)$.
(In fact, we might just write this conjunction if we want to be exceedingly 
careful and clear.)
\\\\
Another equivalent way to say this is ``$A$ is a \emph{proper} subset of $B$.'' 
Alternatively, ``$A$ is \emph{strictly} contained in $B$,'' which means the same thing.
\\\\
Finally, whenever you see a line through an \emph{entire} symbol, it means the
negation of the statement involving that symbol without the line.  So, for example, 
\begin{itemize}
\item $a\notin A$ means that $a$ is not an element of
$A$;
\item $A\not \subset B$ means that $A$ is not a proper subset $B$;
\item $A\nsubseteq B$ means that $A$ is not a subset nor equal to $B$.
\end{itemize}

\end{document}

