%%% Uncomment only one of the next two lines (depending on whether you want the answers printed)
\documentclass[fleqn,addpoints,12pt]{exam}
%%\documentclass[answers,addpoints,12pt]{exam}
% \documentclass[addpoints,12pt]{exam}

\usepackage[usenames, dvipsnames]{color} % defines a new color
%\definecolor{SolutionColor}{rgb}{0.8,0.9,1} % light blue

\renewcommand{\solutiontitle}{\noindent\textbf{Answer: }}

%% How to print correct answer choices:
   \CorrectChoiceEmphasis{\itshape\bfseries} %% <-- bold italics

\pointsinmargin
\pointpoints{pt}{pts}
\marginpointname{pts}

\makeatletter
\newif\ifanswers
\@ifclasswith{exam}{answers}{\answerstrue}{\answersfalse}
\makeatother

\newcommand{\foo}{\ifanswers fooone\else footwo\fi}

%% Change geometry if you want:
%% \usepackage[top=2cm, left=2cm,right=2cm,bottom=1cm]{geometry}%

\usepackage{amsmath}
\usepackage{amsthm,amssymb}
\usepackage{mathtools}
\usepackage{url,multicol,enumerate}
\usepackage{tikz}
\usepackage{comment}

\theoremstyle{remark}
\newtheorem{theorem}{Theorem}
\newtheorem*{prop}{Proposition}
\newtheorem{problem}{Problem}
\newtheorem*{prob}{Problem}
\newtheorem*{answer}{{\bf Answer}}
\newtheorem*{answers}{{\bf Answers}}
\newtheorem*{explanation}{{\bf Explanation}}
\newtheorem*{hint}{{\it Hint}}
\newtheorem*{ex}{Exercise}


%% Some of my own personal favoriate macros... (remove these if you want)

\newcommand{\Ran}{\ensuremath{\operatorname{Ran}}}
\newcommand{\Dom}{\ensuremath{\operatorname{Dom}}}
\newcommand{\Eq}{\ensuremath{\operatorname{Eq}}}
\newcommand{\sF}{\ensuremath{\mathcal{F}}}
\newcommand{\sP}{\ensuremath{\mathcal{P}}}
\newcommand{\R}{\ensuremath{\mathbb R}}
\newcommand{\mr}[1]{\ensuremath{\mathrel{#1}}}
\newcommand{\lra}{\ensuremath{\leftrightarrow}}
\newcommand{\Z}{\ensuremath{\mathbb Z}}
\newcommand{\N}{\ensuremath{\mathbb N}}

\newcommand{\dotsize}{1pt}
\newcommand{\Heq}{\ensuremath{ \; \stackrel{\mathrm{H}}{=}} \; }
\newcommand{\disj}{\ensuremath{\vee}}
\newcommand{\join}{\ensuremath{\vee}}
\newcommand{\conj}{\ensuremath{\wedge}}
\newcommand{\meet}{\ensuremath{\wedge}}
\newcommand{\onlyif}{\ensuremath{\rightarrow}}
\renewcommand{\iff}{\ensuremath{\leftrightarrow}}

% \pagestyle{foot}
%%% Running footer will have a space for page score (if this is not the solution key)
\ifanswers  %% do nothing
\else
\footer{}{}{Score for this page: \makebox[1in]{\hrulefill} out of \pointsonpage{\thepage}}
\fi

\begin{document}

\noindent {\bf MATH 321} 
\hfill {\bf Homework 5 (due 11/2)} 
\hfill  \ifanswers {\bf ANSWERS} \else {\bf NAME:}\phantom{XXXXXXXXXXXXX} \fi 

  \renewcommand{\bigskip}{\vskip1cm}

\pagestyle{foot}

\begin{questions} % Begins the questions environment

% \pagestyle{foot}
  
  %% Ex 1 %%%%%%%%%%%%%%%%%%%%%%%%%%%%%%%%%%%%% 
  \question[2] (V 4.1.2) 
  What are the truth sets of the following statements? List a few elements
  of each truth set.
  \begin{parts}
  \part ``$x$ lives in $y$,'' where $x$ ranges over the set $P$ of all people and $y$
  ranges over the set $C$ of all cities.

  \ifanswers \else \vskip30mm \fi

  \part ``The population of $x$ is $y$,'' where $x$ ranges over the set $C$ of all
  cities and $y$ ranges over $\N$.
  \end{parts}

  \medskip
  \begin{solution}
    \begin{enumerate}[(a)]
    \item 
    \end{enumerate}
  \end{solution}

  \ifanswers \vskip1cm \else \vskip30mm \fi

  %% Ex 2 %%%%%%%%%%%%%%%%%%%%%%%%%%%%%%%%%%%%% 
  \question[2] (V 4.1.10)\\[4pt]
  Prove that for any sets $A$, $S$, $C$, and $D$, if $A \times B$ and $C \times D$ are disjoint,
  then either $A$ and $C$ are disjoint or $B$ and $D$ are disjoint.
  \medskip
  \begin{solution}
  \end{solution}

  \ifanswers \vskip1cm \else \newpage \fi

  %% Ex 3 %%%%%%%%%%%%%%%%%%%%%%%%%%%%%%%%%%%%% 
  \question[3] (V 4.2.9)
  Suppose $R$ and $S$ are relations from $A$ to $B$. Must the following statements
  be true? Justify your answers with proofs or counterexamples.
  \begin{parts}  
    \part  $R \subseteq \Dom(R) \times \Ran(R)$.
    \part  If $R \subseteq S$ then $R^{-1} \subseteq S^{-1}$.
    \part  $(R\cup S)^{-1} = R^{-1}\cup S^{-1}$.
  \end{parts}
  \medskip
  \begin{solution}  \end{solution}
  \ifanswers \vskip1cm \else \newpage \fi


  %% Ex 5 %%%%%%%%%%%%%%%%%%%%%%%%%%%%%%%%%%%%% 
  \question[4] % 4.1.2
  For each of the following binary relations $\rho$ on $\Z$, decide which of the
  given ordered pairs belong to $\rho$. 
  \begin{parts}  
  \part $x\mr{\rho} y \lra x\mr{\mid}y$; $(2,6), (3,5), (8,4), (4,8)$
  \part $x\mr{\rho} y \lra x$ and $y$ are relatively prime; $(5,8), (9,16), (6,8), (8,21)$
  \part $x\mr{\rho} y \lra \mathrm{gcd}(x,y)=7$; $(28,14), (7,7), (10,5), (21,14)$
  \part $x\mr{\rho} y \lra x^2 + y^2 = z^2$ for some integer $z$; $(1,0), (3,9), (2,2), (3,4)$
  \part $x\mr{\rho} y \lra x$ is a Fibonacci number; $(4,3), (7,6), (7,12), (20,20)$
  \end{parts}
  \medskip
 {\it Example:} (a) $\{(2,6),(4,8)\}\subseteq \rho$. 
  (Present your answers in the same format.)\\[4pt] 
{\bf Answers:}
    \begin{enumerate}[(a)]
      \setcounter{enumi}{1}
    \item ~\\[7pt]
    \item ~\\[7pt]
    \item ~\\[7pt]
    \item ~\\[7pt]
    \end{enumerate}
  \begin{solution}  
    \begin{parts}
      \part 
    \end{parts}
  \end{solution}
  % \ifanswers \vskip1cm \else \vskip6cm \fi


  %% Ex 9 %%%%%%%%%%%%%%%%%%%%%%%%%%%%%%%%%%%%% 
  \question[3]
  \label{Ex:9} % 4.1.10
  Let $S = \{0, 1, 2, 4, 6\}$.  Test the following binary
  relations on $S$ for reflexivity, symmetry, antisymmetry, and transitivity.
  \begin{parts}
    \part $\rho = \{ (0,0), (1,1), (2,2), (4,4), (6,6), (0,1), (1,2), (2,4), (4,6)\}$
    \part $\rho = \{ (0,1), (1,0), (2,4), (4,2), (4,6), (6,4)\}$
    \part $\rho = \{ (0,1), (1,2), (0,2), (2,0), (2,1), (1,0), (0,0), (1,1), (2,2)\}$
    % \part $\rho = \{ (0,0), (1,1), (2,2), (4,4), (6,6), (4,6), (6,4)\}$
    \part $\rho = \emptyset$
  \end{parts}
  \medskip
  Example: (a) reflexive, antisymmetric.
  (Present your answers in the same format.)\\[4pt] 
  {\bf Answers:}
    \begin{enumerate}[(a)]
      \setcounter{enumi}{1}
    \item ~\\[7pt]
    \item ~\\[7pt]
    \item ~\\[7pt]
    \end{enumerate}

  \medskip

  \begin{solution} 
  \begin{parts}
    \part 
  \end{parts}
  \end{solution}
  \ifanswers \vskip1cm \else \newpage \fi

  %% Ex 15 %%%%%%%%%%%%%%%%%%%%%%%%%%%%%%%%%%%%% 
  \question[2] % 4.1.17
  Find the reflexive, symmetric, and transitive closure of
  each of the relations in the last two parts of Exercise~\ref{Ex:9}.

  \smallskip

  {\it Example Solution:}
  (a) reflexive closure$(\rho) = \rho$;\\[5pt]
      symmetric closure$(\rho) = \rho \cup \{(1,0), (2,1), (4,2), (6,4)\}$;\\[5pt]
      transitive closure$(\rho) = \rho \cup \{(0,2), (1,4), (2,6), (0,4), (0,6), (1,6)\}$.\\[10pt]
  {\bf Answers:}
    \begin{enumerate}[(a)]
      \setcounter{enumi}{2}
    \item ~\\[2cm]
    \item
    \end{enumerate}

  \begin{solution}   
    \begin{parts} 
      \part 
    \end{parts}
  \end{solution}
  \ifanswers \vskip1cm \else \vskip3cm \fi


  %% Ex 7 %%%%%%%%%%%%%%%%%%%%%%%%%%%%%%%%%%%%% 
  \question[2] (V 4.3.5)
  The following diagram shows two relations $R$ and $S$. Find $S \circ R$.
  \begin{center}
    \includegraphics[height=2.25in]{V-4-3-5}
  \end{center}
  \medskip
  \begin{solution}  \end{solution}
  \ifanswers \vskip1cm \else \newpage \fi



  %% Ex 10 %%%%%%%%%%%%%%%%%%%%%%%%%%%%%%%%%%%%% 
  \question[3] (V 4.4.2)
  In each case, say whether or not $R$ is a partial order on $A$. If so, is it a
  total order?
  \begin{parts}
    \part $A = $ the set of all words of English, 
    $R = \{(x, y) \in A \times A \mid $ the word
      $y$ occurs at least as late in alphabetical order as the word $x\}$.
    \part $A = $ the set of all words of English, 
    $R = \{(x, y) \in A \times A \mid $ the first
      letter of the word $y$ occurs at least as late in the alphabet as the first
      letter of the word $x\}$.
    \part $A = $ the set of all countries in the world, 
    $R = \{(x, y) \in A \times A \mid $ the population of the country $y$ 
      is at least as large as the population of the country $x\}$.
  \end{parts}
  \medskip
  \begin{solution}  \end{solution}
  \ifanswers \vskip1cm \else \newpage \fi



  %% Ex 14 %%%%%%%%%%%%%%%%%%%%%%%%%%%%%%%%%%%%% 
  \question[3] % 4.1.23
  Draw the Hasse diagram for each of the partially ordered sets.
  \begin{parts}
  \item $S = \{a, b, c\}$,\\
    $\rho = \{ (a,a), (b,b), (c,c), (a,b), (b,c), (a,c)\}$.
  \item $S = \{a, b, c, d\}$,\\
    $\rho = \{ (a,a), (b,b), (c,c), (d,d), (a,b), (a,c)\}$.
  \item $S = \{\emptyset, \{a\}, \{a, b\}, \{c\}, \{a, c\}, \{b\}\}$, \\
    $\rho = \; \subseteq$.  That is,
    $A \mr{\rho} B \lra A\subseteq B$.
  \end{parts}
  \medskip
  \begin{solution}
  \end{solution}
  \ifanswers \vskip1cm \else \newpage \fi

\thispagestyle{empty}
\begin{center}
{\bf Recommended Exercises}
\end{center}
The remaining exercises in this assignment are recommended but will not be
graded.  Students are encouraged to solve these problems and ask questions
about them.  We will solve some of them in class.
  %% Ex 16 %%%%%%%%%%%%%%%%%%%%%%%%%%%%%%%%%%%%% 
  \question (V 4.6.2)
  Find the set $\Eq(A)$ of all equivalence relations on the 
  set $A = \{1, 2, 3\}$ and draw the Hasse diagram of the subset 
  inclusion relation on $\Eq(A)$.
  \medskip
  \begin{solution}  \end{solution}
%  \ifanswers \vskip1cm \else \newpage \fi

\bigskip

  %% Ex 4 %%%%%%%%%%%%%%%%%%%%%%%%%%%%%%%%%%%%% 
  \question (V 4.2.10) %\emph{recommended exercise}\\[4pt]
  Suppose $R$ is a relation from $A$ to $B$ and $S$ is a relation from $B$ to $C$.
  Prove that $S \circ R = \emptyset$ iff $\Ran(R)$ and $\Dom(S)$ are disjoint.

  \medskip

  \begin{solution}  \end{solution}
  % \ifanswers \vskip1cm \else \newpage \fi

  \bigskip
  %% Ex 8 %%%%%%%%%%%%%%%%%%%%%%%%%%%%%%%%%%%%% 
  \question (V 4.3.18)
  Suppose $R$ and $S$ are transitive relations on $A$. Prove that if 
  $S \circ R \subseteq R\circ S$, then $R\circ S$ is transitive.
  \medskip
  \begin{solution}  \end{solution}
  % \ifanswers \vskip1cm \else \newpage \fi

  \bigskip

  %% Ex 11 %%%%%%%%%%%%%%%%%%%%%%%%%%%%%%%%%%%%% 
  \question (V 4.4.17)  %\emph{recommended exercise}\\[4pt]
  If a subset of a partially ordered set has exactly one minimal element,
  must that element be a smallest element? Give either a proof or a counter
  example to justify your answer.
  \medskip
  \begin{solution}  \end{solution}
  % \ifanswers \vskip1cm \else \newpage \fi

  \bigskip

  %% Ex 12 %%%%%%%%%%%%%%%%%%%%%%%%%%%%%%%%%%%%% 
  \question (V 4.4.23)
  Prove the following \\[4pt]
  {\bf Theorem 4.4.11}
    Suppose $A$ is a set, $\sF \subseteq \sP(A)$, and $\sF \neq \emptyset$. 
    Then the least upper bound of $\sF$ (in the subset partial order) is 
    $\cup \sF$ and the greatest lower bound of  $\sF$ is  $\cap \sF$.
  \medskip
  \begin{solution}  \end{solution}
  % \ifanswers \vskip1cm \else \newpage \fi

  \bigskip

  %% Ex 17 %%%%%%%%%%%%%%%%%%%%%%%%%%%%%%%%%%%%% 
  \question (V 4.6.19)  % \emph{recommended exercise}\\[4pt]
  Suppose $R$ and $S$ are equivalence relations on a set $A$. Let 
  $T = R\cap S$.
  \begin{parts}
    \part Prove that $T$ is an equivalence relation on $A$.
    \part Prove that for all $x \in A$, $[x]_T = [x]_R \cap [x]_S$.
    % \part Prove that $A/T = (A/R) \cdot (A/S)$. 
    % (See exercise 17 for the meaning of the notation used here.)
  \end{parts}
  \medskip
  \begin{solution}  
    \begin{parts}  \part   \end{parts}
  \end{solution}
\end{questions}
{\large \begin{center} \gradetable[v][pages] \end{center}  }
% {\large \begin{center} \gradetable[v] \end{center}  }
\thispagestyle{empty}

\end{document}
