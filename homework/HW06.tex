%%% Uncomment only one of the next two lines (depending on whether you want the answers printed)
% \documentclass[fleqn,addpoints,12pt]{exam}
\documentclass[answers,addpoints,12pt]{exam}
% \documentclass[addpoints,12pt]{exam}

\usepackage[usenames, dvipsnames]{color} % defines a new color
%\definecolor{SolutionColor}{rgb}{0.8,0.9,1} % light blue

\renewcommand{\solutiontitle}{\noindent\textbf{Answer: }}

%% How to print correct answer choices:
   \CorrectChoiceEmphasis{\itshape\bfseries} %% <-- bold italics

\pointsinmargin
\pointpoints{pt}{pts}
\marginpointname{pts}

\makeatletter
\newif\ifanswers
\@ifclasswith{exam}{answers}{\answerstrue}{\answersfalse}
\makeatother

\newcommand{\foo}{\ifanswers fooone\else footwo\fi}

%% Change geometry if you want:
%% \usepackage[top=2cm, left=2cm,right=2cm,bottom=1cm]{geometry}%

% Create a True False question format
\newcommand*{\TrueFalse}[1]{%
\ifprintanswers
    \ifthenelse{\equal{#1}{T}}{%
        \hspace*{5pt}\textbf{TRUE}\hspace*{14pt}False
    }{
        \hspace*{5pt}True\hspace*{14pt}\textbf{FALSE}
    }
\else
    \hspace*{5pt}{True}\hspace*{16pt}False
\fi
} 
%% The following code is based on an answer by Gonzalo Medina
%% http://tex.stackexchange.com/a/13106/39194
\newlength\TFlengthA
\newlength\TFlengthB
\settowidth\TFlengthA{\hspace*{1in}}
\newcommand\TFQuestion[2]{%
    \setlength\TFlengthB{\linewidth}
    \addtolength\TFlengthB{-\TFlengthA}
    \parbox[t]{\TFlengthB}{#2}
    \parbox[t]{\TFlengthA}{\TrueFalse{#1}}}

\usepackage{amsmath}
\usepackage{amsthm,amssymb}
\usepackage{mathtools}
\usepackage{url,multicol,enumerate}
\usepackage{tikz}
\usepackage{comment}

\theoremstyle{remark}
\newtheorem{theorem}{Theorem}
\newtheorem*{prop}{Proposition}
\newtheorem{problem}{Problem}
\newtheorem*{prob}{Problem}
\newtheorem*{answer}{{\bf Answer}}
\newtheorem*{answers}{{\bf Answers}}
\newtheorem*{explanation}{{\bf Explanation}}
\newtheorem*{hint}{{\it Hint}}
\newtheorem*{ex}{Exercise}


%% Some of my own personal favoriate macros... (remove these if you want)

\newcommand{\Ran}{\ensuremath{\operatorname{Ran}}}
\newcommand{\Dom}{\ensuremath{\operatorname{Dom}}}
\newcommand{\Ker}{\ensuremath{\operatorname{Ker}}}
\newcommand{\Eq}{\ensuremath{\operatorname{Eq}}}
\newcommand{\sF}{\ensuremath{\mathcal{F}}}
\newcommand{\sP}{\ensuremath{\mathcal{P}}}
\newcommand{\R}{\ensuremath{\mathbb R}}
\newcommand{\mr}[1]{\ensuremath{\mathrel{#1}}}
\newcommand{\lra}{\ensuremath{\leftrightarrow}}
\newcommand{\Z}{\ensuremath{\mathbb Z}}
\newcommand{\id}{\ensuremath{\mathrm{id}}}
\newcommand{\N}{\ensuremath{\mathbb N}}

\newcommand{\dotsize}{1pt}
\newcommand{\res}[2]{\ensuremath{#1 \upharpoonright #2}}
\newcommand{\Heq}{\ensuremath{ \; \stackrel{\mathrm{H}}{=}} \; }
\newcommand{\disj}{\ensuremath{\vee}}
\newcommand{\join}{\ensuremath{\vee}}
\newcommand{\conj}{\ensuremath{\wedge}}
% \renewcommand{\or}{\ensuremath{\vee}}
% \renewcommand{\and}{\ensuremath{\wedge}}
\newcommand{\meet}{\ensuremath{\wedge}}
\newcommand{\onlyif}{\ensuremath{\rightarrow}}
\renewcommand{\iff}{\ensuremath{\leftrightarrow}}

% \pagestyle{foot}
%%% Running footer will have a space for page score (if this is not the solution key)
\ifanswers  %% do nothing
\else
\footer{}{}{Score for this page: \makebox[1in]{\hrulefill} out of \pointsonpage{\thepage}}
\fi

\begin{document}

\noindent {\bf MATH 321} 
\hfill {\bf Homework 6 (due 11/16)} 
\hfill  \ifanswers {\bf ANSWERS} \else {\bf NAME:}\phantom{XXXXXXXXXXXXX} \fi 

  \renewcommand{\bigskip}{\vskip1cm}

\pagestyle{foot}

\begin{questions} % Begins the questions environment

% \pagestyle{foot}
  
  %% Ex 1 %%%%%%%%%%%%%%%%%%%%%%%%%%%%%%%%%%%%% 
  \question[3]
  The accompanying figure represents a function with domain $\{4,5,6,7,8\}$.
  \begin{center}
    \includegraphics[height=1.5in]{4-4-1-fig}
  \end{center}
  \begin{parts}
  \part The codomain is \fillin[$\{8,9,10,11\}$]\\[2pt]
  \part The range is \fillin[$\{8,9,10\}$] \\[2pt]
  \part The image of 5 is \fillin[$8$] \\[2pt]
  \part The image of 8 is \fillin[$10$] \\[2pt]
  \part The preimage of 9 is \fillin[$\{6,7\}$]
  \part \TFQuestion{F}{ This function is onto.}
  \part \TFQuestion{F}{ This function is one-to-one.}
  \end{parts}


\vskip5mm

  %% Ex 2 %%%%%%%%%%%%%%%%%%%%%%%%%%%%%%%%%%%%% 
  \question[3] Circle True or False, as appropriate.

  \begin{parts}
    \part \TFQuestion{F}{ A function is onto if and only if every element in the domain has an image. }
    \part \TFQuestion{F}{ A function is onto if and only if every element in the codomain has an image. }
    \part \TFQuestion{T}{ A function is onto if and only if every element in the codomain has a preimage. }
    \part \TFQuestion{F}{ A function is onto if and only if every element in the codomain has a unique preimage. }
    \part \TFQuestion{F}{ A function is onto if and only if $(\text{the range}) \cap (\text{the codomain}) = \emptyset$. }
    \part \TFQuestion{F}{ A function is one-to-one if and only if every element in the codomain has a unique preimage. }
    \part \TFQuestion{T}{ A function is one-to-one if and only if distinct elements in the domain map to distinct elements in the codomain. }
  \end{parts}
  

\newpage

  %% Ex 3 %%%%%%%%%%%%%%%%%%%%%%%%%%%%%%%%%%%%% 
  \question[3]  Let $S = \{0, 2, 4, 6\}$ and $T = \{1, 3, 5, 7\}$.
  Determine whether each of the following sets of ordered pairs is a function with
  domain $S$ and codomain $T$.  If so, it is one-to-one?  Is it onto?
  \begin{parts}
  \part $\{ (0,2), (2,4), (4,6), (6,0)\}$ is \\[4pt]
    \begin{oneparcheckboxes}
      \choice a function
      \choice one-to-one
      \choice onto
      \CorrectChoice none of these
    \end{oneparcheckboxes}

    \vskip5mm
  \part $\{ (6,3), (2,1), (0,3), (4,5)\}$ is \\[4pt]
    \begin{oneparcheckboxes}
      \CorrectChoice a function
      \choice one-to-one
      \choice onto
      \choice none of these
    \end{oneparcheckboxes}

    \vskip5mm
  \part $\{ (2,3), (4,7), (0,1), (6,5)\}$ is \\[4pt]
    \begin{oneparcheckboxes}
      \CorrectChoice a function
      \CorrectChoice one-to-one
      \CorrectChoice onto
      \choice none of these
    \end{oneparcheckboxes}

    \vskip5mm
  \part $\{ (2,1), (4,5), (6,3)\}$ is \\[4pt]
    \begin{oneparcheckboxes}
      \choice a function
      \choice one-to-one
      \choice onto
      \CorrectChoice none of these
    \end{oneparcheckboxes}

    \vskip5mm
  \part $\{ (6,1), (0,3), (4,1), (0,7), (2,5)\}$ is \\[4pt]
    \begin{oneparcheckboxes}
      \choice a function
      \choice one-to-one
      \choice onto
      \CorrectChoice none of these
    \end{oneparcheckboxes}
  \end{parts}
\medskip

\vskip5mm

  %% Ex 4 %%%%%%%%%%%%%%%%%%%%%%%%%%%%%%%%%%%%% 
  \question[3] 
  Let $S = \{a, b, c, d\}$ and $T = \{x, y, z\}$.
  \begin{parts}
  \item  If possible, give an example of a function from $S$ to $T$ that is onto.
    \vfill
  \item  If possible, give an example of a function from $S$ to $T$ that is one-to-one.
    \vfill
  \item  The number of functions from $S$ to $T$ is \fillin[$4^3$].
  \end{parts}

\newpage
  %% Ex 5 %%%%%%%%%%%%%%%%%%%%%%%%%%%%%%%%%%%%% 
  \question  (V 5.1.9) Suppose $f \colon A \to C$ and $g \colon B \to C$ are functions.\\[4pt]
  (See V 5.1.7 for the meaning of notation used in this exercise.)
  \begin{parts}
    \part[2] Prove that if $A$ and $B$ are disjoint, then $f\cup g \colon A \cup B \to C$. 
    \vfill
    \part[1] More generally, prove that 
    $f\cup g \colon A \cup B \to C$ iff $\res{f}{(A\cap B)} = \res{g}{(A\cap B)}$.
    \vskip11cm
  \end{parts}
\newpage

  %% Ex 6 %%%%%%%%%%%%%%%%%%%%%%%%%%%%%%%%%%%%% 
  \question  (V 5.1.17) 
  \begin{parts}
    \part[1]    
    Suppose $g \colon A \to B$ and let $\Ker g = \{(x, y) \in  A \times A \mid g(x) = g(y)\}$.
    Show that $\Ker g$ is an equivalence relation on $A$.

    \vfill
    \part[2] Suppose $R$ is an equivalence relation on $A$ and let $g \colon A \to A/R$ be
    the function defined by the formula $g(x) = [x]_R$. Show that 
    $R= \{(x, y) \in  A \times A \mid g(x) = g(y)\}$.
    \vskip10cm
  \end{parts}


  \newpage
  %% Ex 7 %%%%%%%%%%%%%%%%%%%%%%%%%%%%%%%%%%%%% 
  \question[2]  (V 5.2.16) 
  Suppose $R$ is an equivalence relation on $A$ and suppose $f \colon A \to B$
  is \emph{compatible} with $R$. Ex~5.1.18 gives the definition of compatible, and also asks 
  you to prove that there is a unique function $h \colon A/R \to B$ such that for all $x \in A$, $h([x]_R) = f(x)$.
  Now prove that $h$ is one-to-one iff 
  $(\forall x \in A)(\forall y \in A)(f(x) = f(y) \longrightarrow (x,y) \in R)$.%x\mathrel{R} y)$.


\newpage
\question (cf.~V 5.3.13)
 Suppose $f \colon A \to B$ is onto. Let 
 $K = \Ker(f) = \{(x, y) \in  A \times A \mid f(x) = f(y)\}$.
 By Ex~5.1.17 $K$ is an equivalence relation on $A$.
\begin{parts}
  \part[2]
  Consider the relation $h \subseteq A/K \times B$ 
  defined by $h = \{([x]_K,f(x)) \mid x\in A\}$.
  Prove that $h$ is actually a \emph{function} from $A/K$ to $B$, and satisfies
  $h([x]_K) = f(x)$.
\vfill

  \part[1] Prove that $h$ is one-to-one and onto. 
  (Hint: See Ex~5.2.16)
\vfill


  \part[1] It follows from part (b) that $h$ has an inverse function, 
  $h^{-1} \colon B \to A/K$. \\
  Prove that for all $b \in B$, $h^{-1}(b) = \{x\in A \mid f(x) = b\}$.
\vskip8cm
  % \part  Suppose $g \colon B \to A$. 
  % Prove that $f\circ g = \id_B$ iff $(\forall b \in B)(g(b) \in h(b))$.
\end{parts}

\newpage
  \begin{center}
    {\sc  Suggested Problems}
  \end{center}
  
  The following exercises are also recommended:
  5.1.7, 5.1.18, 5.2.8, 5.2.17, 5.2.18, 5.3.9.
\end{questions}

\vfill
    \ifanswers \else 
    {\large \begin{center} \gradetable[v][pages] \end{center}  }
    \thispagestyle{empty}
    \fi

\end{document}


