%%% Uncomment only one of the next two lines (depending on whether you want the answers printed)
%%\documentclass[fleqn,addpoints,12pt]{exam}
%% \documentclass[answers,addpoints,12pt]{exam}
\documentclass[addpoints,12pt]{exam}

\usepackage[usenames, dvipsnames]{color} % defines a new color
%\definecolor{SolutionColor}{rgb}{0.8,0.9,1} % light blue

\renewcommand{\solutiontitle}{\noindent\textbf{Answer: }}

%% How to print correct answer choices:
   \CorrectChoiceEmphasis{\itshape\bfseries} %% <-- bold italics

\pointsinmargin
\pointpoints{pt}{pts}
\marginpointname{pts}

\makeatletter
\newif\ifanswers
\@ifclasswith{exam}{answers}{\answerstrue}{\answersfalse}
\makeatother
\newcommand{\scratchpage}{%
  \ifanswers % do nothing
  \else \newpage \thispagestyle{empty} \begin{center} -- scratch -- \end{center} \fi}

\newcommand{\foo}{\ifanswers fooone\else footwo\fi}

%% Change geometry if you want:
%% \usepackage[top=2cm, left=2cm,right=2cm,bottom=1cm]{geometry}%

\usepackage{amsmath}
\usepackage{amsthm,amssymb}
\usepackage{mathtools}
\usepackage{url,multicol,enumerate}
\usepackage{tikz}
\usepackage{comment}

\theoremstyle{remark}
\newtheorem{theorem}{Theorem}
\newtheorem*{prop}{Proposition}
\newtheorem{problem}{Problem}
\newtheorem*{prob}{Problem}
\newtheorem*{answer}{{\bf Answer}}
\newtheorem*{answers}{{\bf Answers}}
\newtheorem*{explanation}{{\bf Explanation}}
\newtheorem*{hint}{{\it Hint}}
\newtheorem*{ex}{Exercise}


%% Some of my own personal favoriate macros... (remove these if you want)

\newcommand{\R}{\ensuremath{\mathbb R}}
\newcommand{\Z}{\ensuremath{\mathbb Z}}
\newcommand{\N}{\ensuremath{\mathbb N}}

\newcommand{\dotsize}{1pt}
\newcommand{\Heq}{\ensuremath{ \; \stackrel{\mathrm{H}}{=}} \; }
\newcommand{\disj}{\ensuremath{\vee}}
\newcommand{\join}{\ensuremath{\vee}}
\newcommand{\conj}{\ensuremath{\wedge}}
\newcommand{\meet}{\ensuremath{\wedge}}
\newcommand{\onlyif}{\ensuremath{\rightarrow}}
\renewcommand{\iff}{\ensuremath{\leftrightarrow}}

\pagestyle{foot}
%%% Running footer will have a space for page score (if this is not the solution key)
\ifanswers  %% do nothing
\else
\runningfooter{}{}{Score for this page: \makebox[1in]{\hrulefill} out of \pointsonpage{\thepage}}
\fi

\begin{document}

\noindent {\bf MATH 321} 
\hfill {\bf Homework 7 (due 12/07)} 
\hfill  \ifanswers {\bf ANSWERS} \else {\bf NAME:}\phantom{XXXXXXXXXXXXX} \fi 

  \renewcommand{\bigskip}{\vskip1cm}



\begin{questions} % Begins the questions environment

\pagestyle{foot}
  
\question[3] %% 1
Prove the following identity for all natural numbers $n \geq 1$.
\begin{equation}
  \label{eq:3}
1 + \frac{1}{2} + \frac{1}{4} +\cdots + \frac{1}{2^n} = 2 - \frac{1}{2^n} %% \quad \text{ for $n\geq 1$.}
\end{equation}

%% \begin{solution}
%% (type your solution here and uncomment this and surrounding lines)
%% \end{solution}

\newpage


\question[3] (V 6.1.4) Find a formula for $1+3+5 + \cdots + (2n - 1)$, for
$n \geq 1$, and prove that your formula is correct. ({\it Hint:} First try some particular values of $n$ and
look for a pattern.)
%% \begin{solution}
%% (type your solution here and uncomment this and surrounding lines)
%% \end{solution}

\newpage


\question[3] Use ``strong'' induction (i.e., the 2nd Principle of Induction)
to prove that any amount of postage greater than or equal to 12 cents can be
obtained using only 4-cent and 5-cent stamps.
%% \begin{solution}
%% (type your solution here and uncomment this and surrounding lines)
%% \end{solution}


\newpage
\question[3] In any group of $k$ people, $k\geq 1$, each person is to shake
hands with every other person.  Find a formula for the number of handshakes, and
prove the formula by induction. 
%% \begin{solution}
%% (type your solution here and uncomment this and surrounding lines)
%% \end{solution}
\medskip


\newpage

\question[3] (6.2.3) Suppose $R$ is a total order on a set $A$. Prove that every finite, nonempty
set $B \subseteq A$ has an $R$-smallest element.
%% \begin{solution}
%% (type your solution here and uncomment this and surrounding lines)
%% \end{solution}


\newpage
\question The sequence $1, 1, 2, 3, 5, 8, 13, \dots$ of
\emph{Fibonacci numbers} is defined recursively as follows:
\begin{align*}
  F(0) &= 1, \quad F(1) = 1, \\
  F(n) & = F(n-1) + F(n-2), \quad n\geq 2.
\end{align*}
\begin{parts}
  
\part[2] Prove the following property of the Fibonacci numbers directly (i.e., without using induction):
$(\forall n\geq 6) \; (F(n) = 5F(n-4)+3F(n-5))$.
\vskip8cm
\part[2] Prove the following property of the Fibonacci numbers
using the 2nd Principle of Induction:
$(\forall n\geq 1) \; (F(n) <2^n)$.
\end{parts}
%% \begin{solution}
%% (type your solution here and uncomment this and surrounding lines)
%% \end{solution}
\newpage

\question[3]
Solve the recurrence relation.  That is, give a closed form (i.e., non-recursive)
expression defining $F(n)$.
\begin{enumerate}[(i)]
\item $F(1) = 2$.
\item $F(n) = 2F(n-1) + 2^n$ for $n\geq 2$.
\end{enumerate}
%% \begin{solution}
%% (type your solution here and uncomment this and surrounding lines)
%% \end{solution}

\newpage
\question[3] (6.2.15) What's wrong with the following proof that if
$A \subseteq \N$ and $0 \in A$ then $A = \N$?

\begin{proof}
  We will prove by induction that $(\forall n \in  \N)(n \in A)$.\\[10pt]
\underline{Base case:}
 If $n = 0$, then $n \in A$ by assumption. \\[10pt]
  \underline{Induction step:} Let $n \in \N$ be arbitrary, and suppose that
  $n \in A$. Since $n$ was arbitrary, it follows that every natural number is an
  element of $A$, and therefore in particular $n + 1 \in A$.
   %% So it follows by induction that $(\forall n \in  \N)(n \in A)$.
\end{proof}
%% \begin{solution}
%% (type your solution here and uncomment this and surrounding lines)
%% \end{solution}


\end{questions}

%% {\large \begin{center} \gradetable[v][pages] \end{center}  }
\vfill
{\large \begin{center} \gradetable[h] \end{center}  }
\thispagestyle{empty}

\end{document}

